\documentclass{acm_proc_article-sp}

%% set up bibliography
\usepackage[backend=bibtex,doi=false,isbn=false,url=false]{biblatex}
\bibliography{biblio}  % sigproc.bib is the name of the Bibliography in this case
\usepackage{float}
%\bibliographystyle{acm}

\usepackage{tikz}
\usetikzlibrary{shapes,arrows}

\begin{document}
\tikzstyle{block} = [draw, fill=blue!20, rectangle, 
    minimum height=3em, minimum width=6em]
\tikzstyle{sum} = [draw, fill=blue!20, circle, node distance=1cm]
\tikzstyle{input} = [coordinate]
\tikzstyle{output} = [coordinate]
\tikzstyle{pinstyle} = [pin edge={to-,thin,black}]

\title{State of the Art in \\ {\ttlit Twitter Sentiment Analysis}}


%\subtitle{}
%
% You need the command \numberofauthors to handle the 'placement
% and alignment' of the authors beneath the title.
%
% For aesthetic reasons, we recommend 'three authors at a time'
% i.e. three 'name/affiliation blocks' be placed beneath the title.
%
% NOTE: You are NOT restricted in how many 'rows' of
% "name/affiliations" may appear. We just ask that you restrict
% the number of 'columns' to three.
%
% Because of the available 'opening page real-estate'
% we ask you to refrain from putting more than six authors
% (two rows with three columns) beneath the article title.
% More than six makes the first-page appear very cluttered indeed.
%
% Use the \alignauthor commands to handle the names
% and affiliations for an 'aesthetic maximum' of six authors.
% Add names, affiliations, addresses for
% the seventh etc. author(s) as the argument for the
% \additionalauthors command.
% These 'additional authors' will be output/set for you
% without further effort on your part as the last section in
% the body of your article BEFORE References or any Appendices.

\numberofauthors{5} %  in this sample file, there are a *total*
% of EIGHT authors. SIX appear on the 'first-page' (for formatting
% reasons) and the remaining two appear in the \additionalauthors section.
%
\author{
% You can go ahead and credit any number of authors here,
% e.g. one 'row of three' or two rows (consisting of one row of three
% and a second row of one, two or three).
%
% The command \alignauthor (no curly braces needed) should
% precede each author name, affiliation/snail-mail address and
% e-mail address. Additionally, tag each line of
% affiliation/address with \affaddr, and tag the
% e-mail address with \email.
%
% 1st. author
\alignauthor
Christian B. Hotz-Behofsits\\
       \affaddr{0929002}\\
       \affaddr{christian.hotz-behofsits@tuwien.ac.at}
% 2nd. author
\alignauthor
Thomas Schmidleithner\\
       \affaddr{1025525}\\
       \affaddr{tschmidleithner@auto.tuwien.ac.at}
% 3rd. author
\alignauthor
Dominik Pichler\\
       \affaddr{1026045}\\
       \affaddr{dominik.pichler@aon.at}
\and  % use '\and' if you need 'another row' of author names
% 4th. author
\alignauthor
Matthias Reisinger\\
       \affaddr{1025631}\\
       \affaddr{matthias.reisinger@web0.at}
% 5th. author
\alignauthor
Florian Taus\\
       \affaddr{0627918}\\
       \affaddr{florian.taus@hotmail.com}
}
% There's nothing stopping you putting the seventh, eighth, etc.
% author on the opening page (as the 'third row') but we ask,
% for aesthetic reasons that you place these 'additional authors'
% in the \additional authors block, viz.
%\additionalauthors{Additional authors: John Smith (The Th{\o}rv{\"a}ld Group,
%email: {\texttt{jsmith@affiliation.org}}) and Julius P.~Kumquat
%(The Kumquat Consortium, email: {\texttt{jpkumquat@consortium.net}}).}
%\date{30 July 1999}
% Just remember to make sure that the TOTAL number of authors
% is the number that will appear on the first page PLUS the
% number that will appear in the \additionalauthors section.

\maketitle
\begin{abstract}
The extensive growth of user-generated content in social networks and the common usage of emoticons and hashtags has introduced new possibilities to classify these information. In this paper, an overview of the state-of-the-art regarding sentiment analysis of Twitter messages is provided. The usefulness of existing lexical ressources and special expressions like smileys or hashtags is evaluated and an general introduction to sentiment analysis is given as part of the introduction.
\end{abstract}

% A category with the (minimum) three required fields
%\category{H.4}{Information Systems Applications}{Miscellaneous}
%A category including the fourth, optional field follows...
%\category{D.2.8}{Software Engineering}{Metrics}[complexity measures, performance measures]

%\terms{Theory}

%\keywords{ACM proceedings, \LaTeX, text tagging}

\section{Introduction}


Within the last years Twitter and similar social media platforms grew considerably in terms of user numbers and mainstream fame. Twitter has about 284 Mio. active users, generating approximately 500 Mio. posts\footnote{posts are also known as tweets} on a single day\footnote{\url{https://about.twitter.com/company}, Effective 12.11.2014}. These impressive numbers show the amout of information generated by the crowd. Furthermore Twitter is able to reach a vast number of potential customers especially for consumer businesses. As a result the named services gain more and more acceptance as opinion platforms and powerful tools for both, opinion leader as well as pollster. This trend influenced various marketing and sales strategies and created a new market for companies specialized on sentiment analysis of tweets, like tweetfeel\footnote{\url{www.tweetfeel.com}}, Social Mention\footnote{\url{www.socialmention.com}} and Twitratr\footnote{\url{www.twitratr.com}}.


% whats sentiment analysiss
The aim of sentiment analysis is to determine and try to measure positive and negative feelings, emotions and opinions written in a text. English as language allows to express the same intent in different ways. The main challenge therefore consists in abstracting the intention of the writer from the grammatical and language specific rules. 

% steps
In general, sentiment analysis can be splitted into two steps: a preprocessing and a constitutive sentiment analysis phase. In Twitter sentiment analysis there is another step right before preprocessing: fetching the data from the application programming interface (API). This task is not trivial, because the official API is limited regarding the datasets fetched within certain intervals. This leads to the requirement of a caching mechanism to avoid fetching the same tweets multiple times and allow the usage of more datasets.

\begin{tikzpicture}[auto, node distance=3.3cm,>=latex']
    \node [block] (input) {Twitter};
    \node [block, right of=input, pin={[pinstyle]above:Cache}] (proprocess) {Preprocessing};
    \node [block, right of=proprocess] (analyse) {Analysis};

    \draw [->] (proprocess) -- node[name=u] {$u$} (analyse);
    \draw [->] (input) -- node {$fetch$} (proprocess);
\end{tikzpicture}


% a lot research on how sentiment is sentiment is expressed in other media types social quite new
While sentiment analysis of conventional resources like news papers or articles is quite well investigated the special research area of sentiment analysis in terms of social network posts is relatively unexplored. Especially the possibility of priorizing keywords\footnote{known as hashtags} and the common usage of emoticons lead to advanced possibilities to categorize feelings of posts.


\section{Related Work}
Sentiment analysis a growing and well explored part of Natural Language Processing (NLP). There are several papers from \citeauthor{Pang2008} regarding sentiment analysis in general and related topics like the effects of various machine learning approaches \autocite{Pang2002}\autocite{Pang2008}. Especially the last topic is a well studied field \autocite{Manning2000}.
Some related research areas are document level classification, sentence-level classification and machine learning. Due the limited number of characters within a single twitter post, this topic is quite similar to sentence-level sentiment analysis.
There are several recent papers regarding sentiment analysis in the context of twitter posts. \citeauthor{Agarwal2011} published within \citetitle{Agarwal2011} general information and techniques which cover this area \autocite{Agarwal2011}. In \citetitle{Kouloumpis2011} the utility of linguistic features for detecting the sentiment of tweets are investigated. 
\citeauthor{Saif2012} introduce a novel approach of adding semantics as additional features into the training set. For each extracted entity (e.g. Galaxy S) from Twitter Messages, they add a semantic concept (e.g. Samsung product) as an additional feature, and measure the correlation of the representative concept with negative/positive sentiment \autocite{Saif2012}. In \citetitle{Go2009}, an algorithm is presented, which accurately classifies tweets as positive or negative in respect to a query term\autocite{Go2009}.
\section{Data}
Twitter is a social network and a microblogging service that
allows their users to compose and read so called 
\emph{tweets}. Each tweet 
is a text message and can contain up to 140 characters
including links, pictures\footnote{pictures are replaced 
by a url targeting a public available image resource}, and special tagged words.
These special words start with a well defined prefix 
and add semantic information to the prefixed word.
Table \ref{tab:twitterprefixes} shows an overview of 
common prefixes supported by twitter and their purpose. 
The hashtag \# tags a special word, also known as hashtag. 
It is used to mark important keywords within the tweet.
There is also a at-sign-prefix \emph{@}, which allows 
to mention other twitter users in a post. It's also possible to 
insert urls as part of the message. Regarding the hard message 
length limitation of a tweet, url-shortener are often used to
 insert longer urls.

\begin{table}[H]
\centering
\begin{tabular}{ l | c | r }                 
  prefix & name of character & meaning  \\
  \hline
  \# &  hashtag & keyword, index \\
  @ & at-sign & twitter username \\
  \hline
\end{tabular}
\caption{Common Twitter word prefixes}
\label{tab:twitterprefixes}
\end{table}
\section{Preprocessing}
The Data-Preprocessing-Process is an essential part of sentiment-Analysis. Its goal is to prepare data
for the sentiment analysis and remove unnecessary parts.\autocite{Hemalatha2012}

Unnecessary Parts are:\autocite{Hemalatha2012}
\begin{enumerate}
    \item Remove URLs, Special Characters
    \item Filter Unnecesarry Words
    \item Remove Retweets
\end{enumerate}

Besondere an twitter: Hashtags, Mentions, Smileys
Tokenizen
\section{Methods}

Sentiment analysis of twitter data, or more general, the idea of extracting sentiment and opinions from pieces of text is based on the more general principle of \emph{classification} \cite{Pang2002}. The broad aim of classification is to assign given textual units to a set of classes or categories or the apply some kind of regression or ranking. Sentiment analysis is a specialization of this approach which aims at assigning sentiment values to documents. One application might involve to classify an opinionated text by assigning one of two opposing sentiment polarities, i.e. classifying it as either \emph{positive} or \emph{negative}. This classification task is also referred to as \emph{binary classification} or \emph{sentiment polarity classification} \cite{Pang2002}. But in general, the input to the sentiment classification process is not strictly opinionated which makes this task challenging. Therefore this kind of binary classification might not always be applicable. Different approaches that allow for a more fine grained classification might be appropriate, for example, based on a multi-point scale that allows for more than just two sentiment classes.

The characteristics of twitter messages introduce further challenges. Due to the structure of these messages, sentiment analysis of tweets is different from analysing conventional texts, such as review documents, in various ways. The length of at most 140 characters and the rather informal spelling style pose problems that have to be considered carefully when analysing the data. When preprocessing the texts these aspects already need to be handled so that the actual classification process is supplied with information that is considered useful for the analysis phase. After this step it is necessary to collect the relevant information as so called \emph{features} and organize them into a \emph{feature vector}.

\subsection{Features}

TODO

\subsection{Classifiers}

TODO

\subsubsection*{Na\"{\i}ve Bayes}

TODO

\begin{equation*}
P(c \vert d) = \frac{P(c)P(d \vert c)}{P(d)}
\end{equation*}

\begin{equation*}
P_{\mathrm{NB}}(c \vert d) := \frac{P(c)(\prod^{m}_{i=1}P(f_i \vert c)^{n_i(d)})}{P(d)}
\end{equation*}

\subsubsection*{Maimum Entropy}

TODO

\begin{equation*}
P_{\mathrm{ME}}(c \vert d) := \frac{1}{Z(d)}\mathrm{exp}\left( \sum_{i} \lambda_{i,c} F_{i,c}(d,c) \right)
\end{equation*}

\subsubsection*{Support Vector Machines}

TODO

\subsubsection*{Further techniques}

TODO
\section{Conclusions}


%
% The following two commands are all you need in the
% initial runs of your .tex file to
% produce the bibliography for the citations in your paper.

% You must have a proper ".bib" file
%  and remember to run:
% latex bibtex latex latex
% to resolve all references
%
% ACM needs 'a single self-contained file'!
%
%APPENDICES are optional
%\balancecolumns
\appendix
%Appendix A
\section{Headings in Appendices}



\printbibliography

\end{document}
