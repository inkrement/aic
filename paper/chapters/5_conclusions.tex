\section{Conclusions}
Due to the emerging number of users and Tweets, Twitter sentiment analysis is a growing field. Sentiment analysis itself is a well explored part of NLP, but the analysis of Twitter posts is not as widely investigated as for conventional resources like newspapers or articles. 

However, the analysis of tweets is more like the sentence-level than the paragraph-level analysis due to their short length, which also leads to other problems as the use of slang, emoticons and other uncommon sentence parts. To prepare the data, URLs, unnecessary words, retweets and repeated characters are removed in the preprocessing phase. The use of the right tagger, which ought to be a tagger specialized for Tweets, is important because of the big differences, e.g. TweetNLP has 25 percent less errors than the Stanford POS tagger. 

The relevant information (``features'') is then determined and classified with the support vector machine or the Na\"ive Bayes method. The Na\"ive Bayes method classifies documents based on the Bayes' rule while support vector machines are trained to create a hyperplane that separates the two classes (``positive'' and ``negative'' in sentiment classification), so that Tweets can be classified as above or below this hyperplane.

We saw that Twitter sentiment analysis is an important research field but needs some more work to improve the quality and accuracy of both preprocessing and classifying the content of a tweet.
