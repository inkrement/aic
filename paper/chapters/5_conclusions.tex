\section{Conclusions}
Due to the growing number of users and Tweets, Twitter sentiment analsyis is a growing field. Sentiment analysis itself is a well explored part of NLP, but the analsysis of Twitter posts is not as bridely investigated as for conventional resources like news papers or articles. 

However, the analysis of tweets is more like the sentence-level but the paragraph-level analsyis due to their short length, which also leads to other problems as the use of slang, emoticons and other uncommon sentence parts. To prepare the data, the preprocessing phase to remove URLs, unnecessary words, retweets and repeated characters is an important part for the analsysis. For this task, several tokenizers and taggers are available. The relevant information ("features") is then determined and classified with the support vector machine or the Na\"ive Bayes method.

We saw that Twitter sentiment analsis is an important research field but needs some more works to improve the quality and accuracy of both preprocessing and classifying the content of a tweet.
