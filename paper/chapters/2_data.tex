\section{Data}
Twitter is a social network and microblogging service that
allows their users to compose and read so called 
\emph{tweets}. The user registration is not limited
to professional writers, authors or similar expert groups 
and there are no quality safeguards or writing guidelines. 

Each tweet is a text message and can contain up to 140 characters
including links, pictures\footnote{pictures are replaced 
by a url targeting a public available image resource}, and special tagged words.
These special tagged words start with a well defined prefix 
and add semantic information to the prefixed string.
Table \ref{tab:twitterprefixes} shows an overview of 
common prefixes supported by Twitter and their purpose. 
A hash (\#) labels a word, also known as hashtag.
It is used to mark important keywords within tweets.
There is also a at-sign-prefix \emph{@}, which allows 
to mention and link other Twitter users in a post. Tweet composers can 
insert URLs as part of their messages. Regarding the hard 
length limitation of a tweet, url-shortener are often used to
insert longer internet addresses.

\begin{table}[H]
\centering
\begin{tabular}{ l | c | r }                 
  prefix & name of character & meaning  \\
  \hline
  \# &  hashtag & keyword, index \\
  @ & at-sign & twitter username \\
  \hline
\end{tabular}
\caption{Common Twitter word prefixes}
\label{tab:twitterprefixes}
\end{table}

The combination of the 140 character restriction and
the missing writing guidelines leads to another problem: 
Slang, emoticons and punctuation that are uncommon in longer 
text messages are heavily used.\autocite{davies2011}
In terms of sentiment analysis, tweet processing is more comparable to
sentence level analysis as paragraph analysis. 

%These circumstances make it even harder to  ...