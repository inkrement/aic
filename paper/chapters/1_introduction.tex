\section{Introduction}


Within the last years Twitter and similar social media platforms grew considerably in terms of user numbers and mainstream fame. Twitter has about 284 Mio. active users, generating approximately 500 Mio. posts\footnote{posts are also known as tweets} on a single day\footnote{\url{https://about.twitter.com/company}, Effective 12.11.2014}. These impressive numbers show the amout of information generated by the crowd. Furthermore Twitter is able to reach a vast number of potential customers especially for consumer businesses. As a result the named services gain more and more acceptance as opinion platforms and powerful tools for both, opinion leader as well as pollster. This trend influenced various marketing and sales strategies and created a new market for companies specialized on sentiment analysis of tweets, like tweetfeel\footnote{\url{www.tweetfeel.com}}, Social Mention\footnote{\url{www.socialmention.com}} and Twitratr\footnote{\url{www.twitratr.com}}.


% whats sentiment analysiss
The aim of sentiment analysis is to determine and try to measure positive and negative feelings, emotions and opinions written in a text. English as language allows to express the same intent in different ways. The main challenge therefore consists in abstracting the intention of the writer from the grammatical and language specific rules. 

% steps
In general sentiment analysis can be splitted into two steps: a preprocessing and a constitutive sentiment analysis phase. In Twitter sentiment analysis there is another step right bevor preprocessing: fetching the data from the application programming interface (API). This task is not trivial, because the official API is limited regarding the datasets fetched within certain intervals. This leads to the requirement of a caching mechanism to avoid fetching the same tweets multiple times and allow the usage of more datasets.

\begin{tikzpicture}[auto, node distance=3.3cm,>=latex']
    \node [block] (input) {Twitter};
    \node [block, right of=input, pin={[pinstyle]above:Cache}] (proprocess) {Preprocessing};
    \node [block, right of=proprocess] (analyse) {Analysis};

    \draw [->] (proprocess) -- node[name=u] {$u$} (analyse);
    \draw [->] (input) -- node {$fetch$} (proprocess);
\end{tikzpicture}


% a lot research on how sentiment is sentiment is expressed in other media types social quite new
While sentiment analysis of conventional resources like news papers or articles is quite well investigated the special research area of sentiment analysis in terms of social network posts is relatively unexplored. Especially the possibility of priorizing keywords\footnote{known as hashtags} and the common usage of emoticons lead to advanced possibilities to categorize feelings of posts.


\section{Related Work}
Sentiment analysis a growing and well explored part of Natural Language Processing (NLP). There are several papers from \citeauthor{Pang2008} regarding sentiment analysis in general and related topics like the effects of various machine learning approaches \autocite{Pang2002}\autocite{Pang2008}. Especially the last topic is a well studied field \autocite{Manning2000}.
Some related research areas are document level classification, sentence-level classification and machine learning. Due the limited number of characters within a single twitter post, this topic is quite similar to sentence-level sentiment analysis.
There are several recent papers regarding sentiment analysis in the context of twitter posts. \citeauthor{Agarwal2011} published within \citetitle{Agarwal2011} general information and techniques which cover this area \autocite{Agarwal2011}. In \citetitle{Kouloumpis2011} the utility of linguistic features for detecting the sentiment of tweets are investigated. 
\citeauthor{Saif2012} introduce a novel approach of adding semantics as additional features into the training set. For each extracted entity (e.g. Galaxy S) from Twitter Messages, they add a semantic concept (e.g. Samsung product) as an additional feature, and measure the correlation of the representative concept with negative/positive sentiment \autocite{Saif2012}. In \citetitle{Go2009} a algorithm is presented, which accurately classifies tweets as positive or negative, in respect to a query term\autocite{Go2009}.