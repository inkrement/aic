%%%%%%%%%%%%%%%%%%%%%%%%%%%%%%%%%%%%%%%%%%%%%%
% Head matter - can we try to be consistent on
% included packages
\documentclass{beamer}
\mode<presentation>
{\usetheme{default}
 \usecolortheme{default}
 \usefonttheme{default}
 \setbeamertemplate{navigation symbols}{}
 \setbeamertemplate{caption}[numbered]} 
\usepackage[english]{babel}
\usepackage[utf8x]{inputenc}
\usepackage{graphicx}
%%%%%%%%%%%%%%%%%%%%%%%%%%%%%%%%%%%%%%%%%%%%%%
% Formatting for title page
\title[Twitter Sentiment Analysis]{Twitter Sentiment Analysis}
\author{Christian Hotz-Behofsits, Dominik Pichler, Matthias Reisinger, Thomas Schmidleithner, Florian Taus}
\institute{Advanced Internet Computing}
\date{5th December 2014}
%%%%%%%%%%%%%%%%%%%%%%%%%%%%%%%%%%%%%%%%%%%%%%
\begin{document}
\begin{frame}
  \titlepage
\end{frame}
%\section{Introduction}
% There is actually no point to the sections 
% since we are not using table of contents mark up
\begin{frame}{Introduction}
\begin{block}{Aim}
Aggregate sentiment values of different tweets by a value specific keyword via the Twitter API.
\end{block}
\begin{itemize}
   \item Load Tweets from Twitter API
   \item Preprocessing
   \item Classification
   \item Aggregation of overall sentiment
\end{itemize}
\end{frame}

\begin{frame}{Twitter API}
Twitter offers several APIs for different purposes. All require authorization and a valid Twitter API-account.

\begin{itemize}
	\item REST API
	\begin{itemize}
    	\item limited to 180 search queries/15 minutes
        \item limited to 15 fetches/15 minutes (100 tweets each)
        \item fetch all tweets at once
    \end{itemize}
    \item Streaming API
	\begin{itemize}
    	\item no limitation regarding tweets per minute
        \item continuous tweet-stream
    \end{itemize}
\end{itemize}
\end{frame}

% Commands to include a figure:
%\begin{figure}
%\includegraphics[width=\textwidth]{your-figure's-file-name}
%\caption{\label{fig:your-figure}Caption goes here.}
%\end{figure}

\begin{frame}{Preprocessing}
\begin{block}{}
Preprocess a tweet message and extract important information (feature vector) for classification.
\end{block}
\begin{itemize}
	\item[Tokenizer]
    splits input into related parts (tokens). e.g. Stanford NLP PTBTokenizer, Carnegie Mellon Twokenizer
	\item[Tagger]
    adds information to each token (e.g. word-class).  e.g Stanford NLP MaxentTagger
	\item[Filter token list] 
    filtering based on tagged-information. 
\end{itemize}


\end{frame}

\begin{frame}{Classification}
%\begin{block}{}
%Classification aims at dividing inputs into at least two classes by the use of a learned model.
%\end{block}
\begin{itemize}
	\item Aims at dividing inputs into at least two classes
    \item In this course: assign sentiments to textual inputs
    \item Supervised and unsupervised approaches
    \item Common supervised algorithms: Support vector machines (SVM), naive bayes, \ldots
    \begin{itemize}
        \item Textual input in form of a \emph{feature vector}
        \item The feature vector contains relevant words such as verbs, adjectives and hashtags
    \end{itemize}
\end{itemize}
\end{frame}


\begin{frame}{Classification (Implementation)}
\begin{itemize}
	\item Based on WEKA
    \item SVM as classifier (LibSVM)
    \item Train model with stanford twitter test corpus\footnote{\tiny{http://help.sentiment140.com/for-students}} (manually reduced to 1000 positive and 1000 negative instances)
	\item Evaluation with testset (359 instances):\\ \textbf{64.35 \%} classified correctly
\end{itemize}
\end{frame}

\begin{frame}{Aggregate Classified Sentiments}
Calcuate the weighted sentiment for each tweet whereby more recent dates are weighted heavier
\begin{itemize}
	\item The weight of the most recent positive tweet is 1
    \item The weight of the oldest is 0.8
    \item The weights in between decrease linear
    \item $\sum_{i = 0}^{\#tweets-1}1-i*\frac{decreaseFactor}{\#tweets-1}$
    \item Just positive tweets are weighted
\end{itemize}
\end{frame}


\begin{frame}{Prototype}
Some libraries and attributes of our prototype.

\begin{itemize}
	\item Spring Boot
    \item REST Backend
    \item AngularJS Frontend
    \item PTBTokenizer and Twokenizer
    \item Maxent Tagger
    \item Weka
\end{itemize}

\begin{block}{}
$\rightarrow$ live demo!
\end{block}

\begin{block}{}
Code available at: https://github.com/inkrement/aic
\end{block}

\end{frame}

\end{document}

